\documentclass[aps,pra,notitlepage,amsmath,amssymb,letterpaper,12pt]{revtex4-1}
\usepackage{amsthm}
\usepackage{graphicx}
%  Above uses the Americal Physical Society template for Physical Review A
%  as a reasonable and fully-featured default template
 
%  Below define helpful commands to set up problem environments easily
\newenvironment{problem}[2][Problem]{\begin{trivlist}
\item[\hskip \labelsep {\bfseries #1}\hskip \labelsep {\bfseries #2.}]}{\end{trivlist}}
\newenvironment{solution}{\begin{proof}[Solution]}{\end{proof}}
 
% --------------------------------------------------------------
%                   Document Begins Here
% --------------------------------------------------------------
 
\begin{document}
 
\title{A Brief Review of Calculus}
\author{Logan Gantner \\[-7pt] Tim Frenzel}
\affiliation{CS 510, Computational and Data Sciences, Chapman University}
\date{\today}

\maketitle

\section{The Derivative} % Specify main sections this way


\noindent
Given a continuous function $f(x)$, we frequently are interested in considering what is known as its \emph{derivative}. In simple terms, the derivative of $f$ at point $x$ is the rate at which the function increases or decreases for very small (infinitesimal) changes in $x$. This is known as the instantaneous rate of change. Not all functions have well-defined values for the instantaneous rate of change--they may be discontinuous at $x$, or contain corners or cusps. When a function does have a well-defined derivative for all values of $x$ within an open set, we say that $f$ is \emph{differentiable} on this set. When a function is differentiable on an open set containing $x$, we define its derivative at $x$, $f'(x)$, as follows:

\begin{equation}
f'(x) = \lim_{h \rightarrow \infty} \frac{f(x+h) - f(x)}{h} \,,
\end{equation}

\noindent
that is, the limit of the slope of the secant line connecting f(x) to f(x+h) for smaller and smaller values of $h$. The function fails to be differentiable when the limit does not exist.
 
% Repeat as needed
 
\end{document}